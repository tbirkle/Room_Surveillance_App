%
% Praxissemester Bericht
% Tobias Birkle
% Vorlage: Sven Hodapp



% -----------
% 1. Präambel
% -----------


% Allgemeine Einstellungen
% ------------------------
\documentclass[
	pdftex,%              PDFTex verwenden da wir ausschliesslich ein PDF erzeugen.
	a4paper,%             Wir verwenden A4 Papier.
	oneside,%             Einseitiger Druck.
	12pt,%                Grosse Schrift, besser geeignet für A4.
	halfparskip,%         Halbe Zeile Abstand zwischen Absätzen.
	%chapterprefix,%       Kapitel mit 'Kapitel' anschreiben.
	headsepline,%         Linie nach Kopfzeile.
	footsepline,%         Linie vor Fusszeile.
	bibtotocnumbered,%    Literaturverzeichnis im Inhaltsverzeichnis nummeriert einfügen.
	idxtotoc%             Index ins Inhaltsverzeichnis einfügen.
]{article}

\usepackage[a4paper, includeheadfoot, bindingoffset=0cm, 
left=2cm, right=2cm, top=2cm, bottom=2 cm]{geometry}

\usepackage{fancyhdr}
\pagestyle{fancy} 
\renewcommand{\headrulewidth}{0pt}


\usepackage[utf8]{inputenc}
\usepackage[german]{babel}   % deutsche Silbentrennung
\selectlanguage{german}   % damit Table Of Contents Inhaltsverzeichnis genannt wird

\usepackage{geometry}   % Seitenränder einstellbar
\usepackage{textcomp}   % Sonderzeichen, wie Eurosymbol


% Absätze mit Leerzeile ohne Einrückung
\newenvironment{absaetze}
{\addtolength{\parskip}{\baselineskip}\parindent 0pt}
{}


% Bilder, Farben, farbige Tabellen
% --------------------------------
\usepackage{graphicx, color, colortbl}
\usepackage{array}       % Erweiterte Tabelleneigenschaften.
%\usepackage{floatflt}   % Bild kann von Text umflossen werden.



% Palatino Schrift
% ----------------
%\usepackage[T1]{fontenc}
%\usepackage[osf]{mathpazo}   % osf aktiviert Mediävalziffern/Minuskelziffern



% Sonstige Pakete
% ---------------
%\usepackage{anysize}   % Seitenränder verändern
%\usepackage{setspace}   % 1.5em Zeilenabstand \begin{onehalfspacing}
%\usepackage{bibgerm}   % Anzeigestil des Literaturverzeichnis (gerabbrv)
\usepackage{listings}  \lstset{numbers=left, numberstyle=\tiny, numbersep=20pt} % Programmcode einfügen
\usepackage{subfigure}


\renewcommand{\baselinestretch}{1.1}\normalsize
% PDF Eigenschaften
% -----------------
\usepackage
[
	colorlinks=false,
	bookmarks = true,
	pdftitle={Room surveillance app},
	pdfsubject={Documentation},
	pdfkeywords={Documentation, Ubicom, Room surveillance},
	pdfauthor={Benjamin Kugler, Tobias Birkle},
	urlcolor=blue,
	pdfstartview=FitH
]{hyperref}



\usepackage{float}

% --------------------
% 2. Dokumenten Anfang
% --------------------

\begin{document}


% Deckblatt
% ---------
\begin{absaetze}
\begin{titlepage}

	\includegraphics[width=1.00\textwidth]{images/HTWG_logo.PNG}

	\vspace*{1cm}
	\begin{center}
		\Huge
		\vspace{2cm}
		\textbf{Area surveillance system} \\
		\vspace{1cm}
		\large
		
		\vspace{1cm}
		\textbf{Benjamin Kugler, 286346} \\
		\textbf{Tobias Birkle, 286081} \\
		\today\\
		\vspace{2cm}
		
		
		Prof. Dr. rer. nat. Ralf E.D. Seepold\\
		

		
	\end{center}
	\normalsize
	\vfill
\end{titlepage}






\end{absaetze}

\begin{absaetze}
\renewcommand{\thepage}{\arabic{page}}
\fancyfoot[CE, CO]{}

\fancyfoot[RE, RO]{\thepage}
\fancyhead[]{}


\renewcommand{\thepage}{\Roman{page}}
%\addcontentsline{toc}{section}{Kurzfassung}
\end{absaetze}
% Inhaltsverzeichnis anzeigen
% ---------------------------
\setcounter{page}{1}
%\tableofcontents
%\newpage
%\listoffigures % Abbildungsverzeichnis
%\listoftables




% ---------
% 3. Inhalt
% ---------

\begin{absaetze}
\section{Project description}
The \emph{room surveillance app} observes rooms in your building. Whenever the system is active, it will take a picture if a movement is detected. Afterwards the picture will be sent to your smart phone.


\subsection{System architecture}
The movements will be detected with a passive infra-red-sensor (PIR). The PIR is connected to an arduino board. The arduino and the camera are connected to a LAN. If the PIR is triggered, the arduino gets an image from the camera over the common gateway interface(CGI). The image will be uploaded to a webserver. The smart phone app polls in a specific interval, whether a new picture is available on the webserver and downloads it.






\end{absaetze}

% Literaturverzeichnis
% http://ctan.mackichan.com/macros/latex/exptl/biblatex/doc/biblatex.pdf - 2.1.1 Regular Types
%\addcontentsline{toc}{section}{Literatur}
%\bibliographystyle{gerabbrv}
%\bibliographystyle{alpha}
%\bibliographystyle{plain}
%\bibliography{bib/references}

\renewcommand{\thepage}{}
\end{document}


